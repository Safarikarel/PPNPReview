\section{Event characterization}
\label{secks:eventchar}

The first analysis of LHC heavy-ion data revealed the charged particle density and the energy density achieved in Pb--Pb interactions at an unprecedent collision energy of $\sqrt{s_{\rm NN}} = 2.76$~TeV per nucleon pair in centre-of-mass system. The estimated values, as well as results of practically all other measurements, strongly depend on the geometry of the collision (also called centrality), more precisely on the distance $b$ of the centres of colliding nuclei in the plane transverse to the beam axis, called impact parameter of the collision. The impact parameter determines the volume of the interaction region, i.e. how violent the collision was.
%how many nucleons from the incoming nuclei actually took part in the collision
Therefore in this Section we first describe how the centrality of Pb--Pb collisions is determined, and then we turn to the basic measurements which characterize these interactions at the LHC.



\subsection{Centrality determination}
\label{subsecks:centrality}

The lead nuclei are relatively extended objects, their size is about 14~fm. To classify events according what part of the two nuclei participated in the interaction, the concept of collision centrality is commonly introduced in the field of heavy-ion physics. The centrality of the collision is related to its geometrical parameters, such as the impact parameter $b$, for example. These parameters are inferred by comparison of experimental data with simulations of interactions. In this context the geometrical Glauber model is typically used~\cite{Miller:2007ri}, based on R.J.~Glauber description of pA and A--A scattering~\cite{Glauber:1955qq,Glauber:2006gd}. For the event simulation the Monte Carlo implementation of Glauber model is used~\cite{Shor:1988vk,Alver:2008aq}, which is realized by the following steps:
\begin{itemize}
    \item{randomly sample the position of each nucleon inside nucleus according Woods--Saxon distribution (two-parameter Fermi distribution), using the parameters from the analysis of low-energy elastic e--A scattering~\cite{DeJager:1987qc};}
    \item{randomly sample the collision impact parameter $b$ with probability distribution $P(b) \propto b {\rm d}b$ (up to $b_{\rm max} = 20$~fm, i.e. well above the lead nucleus diameter);}
    \item{assuming nucleons are moving along straight lines parallel to the beam direction, the pair of nucleons collides if their centres are closer than $\sqrt{\sigma_{\rm NN}/\pi}$ in the transverse plane, where $\sigma_{\rm NN} = (64 \pm 5)$~mb is the inelastic nucleon--nucleon cross section, estimated from LHC pp measurements;}
\end{itemize}
and for each event the number of nucleons participating in at least one collision ($N_{\rm part}$) and the number of these binary collisions ($N_{\rm coll}$) are counted. Then the total nuclear Pb--Pb cross section ($\sigma_{\rm PbPb}$) is calculated as the fraction of $\pi b_{\rm max}^2$ given by the ratio of the number of events with $N_{\rm coll} \geq 1$ to the number of all generated events. The cross section for collisions with the impact parameter in the interval $(0, b)$ is obtained the same way, counting the events with $N_{\rm coll} \geq 1$ and the impact parameter within that interval. The centrality for this impact-parameter selection is its cross section expressed as the percentage of $\sigma_{\rm PbPb}$. A centrality class is defined by its lower and upper percentages, corresponding to the events within impact-parameter interval $(b_{\rm l}, b_{\rm u})$, where the lower percentage is the part of $\sigma_{\rm PbPb}$ up to the impact parameter $b_{\rm l}$ and the upper percentage is that part up to $b_{\rm u}$. Other characterizations of centrality classes, such as the mean number of participants $\langle N_{\rm part} \rangle$ and the mean number of binary collisions $\langle N_{\rm coll} \rangle$ (obtained as the average values for events within that class) are often needed. For completeness, the geometrical overlap function (integral of the product of the two transverse nuclear densities in the overlapping region) in the Monte Carlo formulation of Glauber model is defined as $T_{\rm AA} = N_{\rm coll} / \sigma_{\rm NN}$.

However, none of the geometrical quantities mentioned above ($b$, $N_{\rm part}$, $N_{\rm coll}$) is directly measurable in an experiment. Therefore, an experimental observable, which strongly correlates with the collision impact parameter, has to be used to classify the events according their centrality. For example, charged-particle multiplicity $N_{\rm ch}$ within some pseudo-rapidity region covered by a detector is often used. Then the centrality selection of events within an impact-parameter interval is replaced by the selection using a $N_{\rm ch}$ interval. In an ideal case, when one would be able to measure the event distribution in such new selection variable for all Pb--Pb nuclear collisions, it will be possible to define centrality selection and centrality percentiles without a model, using only this distribution and its integral. But for very peripheral collisions (large $b$, low $N_{\rm ch}$) the experimental event sample is contaminated by electromagnetic interactions, these processes have at LHC energies a huge cross section (more than two orders of magnitude larger than nuclear cross section) and contribute to low multiplicity events. It is necessary to suppressed them, at least partly, already during the data taking (trigger on a minimal multiplicity, or requiring some signal in ZDC's), which inevitably makes event trigger less efficient for very peripheral collisions. For these reasons, the event distribution in a variable such as $N_{\rm ch}$ is usable for centrality selection starting above some value, typically excluding peripheral collisions corresponding to the centrality class 90--100\,\%, where the contamination and the trigger inefficiency cannot be neglected. In order to find the value from where the distribution can be used and to relate this so-called anchor point to the centrality, the Glauber Monte Carlo needs to be supplemented with a model of particle production. Such model also allows to calculate for the experimental centrality selection corresponding $\langle N_{\rm part} \rangle$ and $\langle N_{\rm coll} \rangle$, taking into account the finite resolution of the selection variable $N_{\rm ch}$ with respect to the collision impact parameter $b$.



%example of centrality selection plot
\subsection{Charged particle density}
\label{subsecks:partdensity}
%charged particle density vs energy
%charged particle density vs centrality
%charged particle density vs pseudorapidity
\subsection{Energy density}
\label{subsecks:energydensity}
%energy density vs energy
