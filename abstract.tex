\begin{abstract}
A new era in the study of high-energy nuclear collisions began when the
CERN Large Hadron Collider (LHC) provided the first collisions of lead nuclei
in late 2010. In the first three years of operation the ALICE, ATLAS and CMS
experiments each collected \PbPb\ data samples of more than 150~\mubinv\ at
$\rootsNN = 2.76$\TeV, exceeding the previously studied collision energies
by more than an order of magnitude. These data have provided new insights into
the properties of QCD matter under extreme conditions, with extensive
measurements of soft particle production and newly accessible
hard probes of the hot and dense medium. In this review, we provide
a comprehensive overview of the results obtained in heavy-ion collisions
at the LHC so far, with particular emphasis on the complementary nature
of the observations by the three experiments.
In particular, the combination of ALICE's strengths at hadron identification,
the strengths of ATLAS and CMS to make precise measurements of high \pT\
probes, and the resourceful measurements of collective flow by all of the
experiments have provided a rich and diverse dataset in only a few years.
While the basic paradigm established at RHIC --- that of a hot, dense medium that flows with a viscosity to
shear-entropy ratio near the predicted lower bound, and which degrades the energy
of probes, such as jets, heavy-flavours and J$/\psi$ --- is confirmed at the LHC,
the new data suggest many new avenues for extracting its properties in detail.
\end{abstract}
