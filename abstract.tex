\begin{abstract} 
A new era in the study of high-energy nuclear collisions began when the 
CERN Large Hadron Collider (LHC) provided the first collisions of lead nuclei
in late 2010. In the first three years of operation the ALICE, ATLAS and CMS 
experiments each collected \PbPb\ data samples of more than 150\mubinv at 
$\rootsNN = 2.76$\TeV, exceeding the previously studied collision energies 
by more than an order of magnitude. These data provided new insights into
the properties of QCD matter under extreme conditions, with extensive
measurements of soft particle production and newly accessible probes
of hard probes of the hot and dense medium. In this reviwe, we provide
a comprehensive overview of the results obtained in heavy-ion collisions
at the LHC so far, with particular emphasis on the complementary nature
of the observations by the three experiments.
\end{abstract}
