\section{Introduction}
\label{secall:intro}
\subsection{Physics motivation}
\label{subsecall:motivation}
\subsection{LHC heavy-ion running}
\label{subsecall:running}
\subsection{Detectors at the LHC}
\label{subsecall:detectors}

%ATLAS
The ATLAS detector is one of the general-purpose particle physics
detectors at the LHC.  
It has three main detector systems: the inner detector (ID), the calorimeter,
and the muon spectrometer (MS).
%
The inner detector tracks charged particles using three separate detector
technologies, and the spectrometer is immersed in a 2T axial field from a 
superconducting solenoid magnet 1.2m from the nominal beam axis.
The pixel detector typically provides three high-resolution space points
with three layers of pixel detector surrounding the beam pipe within
$|z|<400 mm$ (covering approximately $|\eta|<2$ and 4 disks at forward
angles covering out to $|\eta|<2.5$.
The semiconductor tracker detectors are silicon strips covering out to
$|\eta|<X$ in the barrel region and $|\eta|<2.5$ in the forward regions.
The transition-radiation tracker covers $|\eta|<2$ with straw tubes in
both the barrel and forward regions.
%
The ATLAS calorimeter has large coverage in pseudorapidity ($|\eta|<4.9$)
and longitudinal segmentation in both electromagnetic and hadronic
sections.
In the barrel region, the electromagnetic calorimeter has three
layers and a presampler layer, all using liquid argon technology.  
The first layer has very high resolution
in the $\eta$ direction, allowing discrimination of photons from
neutral hadron decays.
The second layer is coarser but deeper, providing the primary energy
measurement for electromagnetically interacting particles (photons and
electrons), while the third layer is there to catch the tails of the
deposited electromagnetic showers.
%
The ATLAS hadronic calorimeter uses steel absorber and measures the
hadronic showers by means of scintillating tiles.
In the endcap region, beyond $|\eta|=2$, 
both hadronic and electromagnetic sections use LAr technology with
relativly coarse cell segmentation.
The ATLAS forward calorimeter (FCal) covers $|\eta|=3.2-4.9$, using
a matrix of copper and liquid argon in the electromagnetic section,
and tungsten and liquid argon for the hadronic section.
%
The ATLAS muon spectrometer covers $|\eta|<2.7$ with precision drift
chambers
in the barrel region and cathode strip chambers in the forward region.
The bending of muons is provided by three air-core toroids, giving
a momentum resolution ranging from approximately 2\% up to about
10\% at $\pT=1$ TeV.
%

ATLAS provides a sophisticated multi-level trigger system for 
selection of physics objects (jets, taus, photons, electrons, muons,
and missing transverse energy).
Jet triggering is done both seeding on energy deposited into localized
regions of the calorimeter, as well as a full reconstruction of the jets
using a similar algorithm as used in the offline analysis.
Electron and photon triggering uses smaller regions in the calorimeter
than for jets, and also applies selections based on the measured shower
shape and leakage in the hadronic sections.
Muon triggering is provided by thin-gap chambers and resistive plate chambers,
covering about 90\% of the solid angle out to $|\eta|=2.7$.
In general, triggering on tau leptons and missing transverse energy (e.g.
from $W$ bosons) is not utilized for any heavy ion analysis, as both
are highly contaminated by the large fluctuations in the 
underlying event.
