\section{Flow correlations}
\label{sec:ps:flow}
%5. Flow Correlations & Flow & PID Flow (PS)
%- Geometric fluctuations and origin of v_n vs ecc_n (Glauber, v3)
%- non-PID flow (ALICE first results)
%- PID flow (ALICE v2 scaling)
%- 2PC (CMS v_n?)
%- Event plane (ATLAS EP)
%- Explaining the ridge (2PC from 3 experiments)
%- Flow fluctuations
%- Event plane correlations
%- Success of viscous hydro
%- Comparison with RHIC
%- Flow in p+Pb (ridge, double ridge, PID)
%- Open questions
\subsection{Introduction}
While the previous section covered physical mechanisms which induce
correlations between multiple hadrons, this section covers the
phenomenon of ``collective flow'', which leads to the correlations of
essentially all the particles in every event.
This results from the translation of anisotropies in the initial shape of the
colliding nuclei into anisotropies in momentum space, something that
would not occur if individual nucleon-nucleon collisions emitted independently
of each other.
The characterization of a ``shape'' in a final state particle distribution
is typically performed using a Fourier decomposition of the azimuthual
angle distrbution of final state particles.
Of course, averaging over an ensemble of independent events would lead to
the observation of no net anisotropy.  Thus, the presence of harmonic
oscillations in the final state requires the estimation of an ``event plane''
from the particle themselves, with an axis that points in the direction of
the largest momentum flow.

While this phenomenon was observed decades ago in the collisions of large
nuclei at low energies, this was straightforward to understand as the
reinteraction of the initial baryons and the produced hadrons, which
would thermalize and evolve as a ``hadron gas''.  
However, its persistence at higher energies, particularly at RHIC where the
value of $v_2$ averaged over \pT was twice as large as it was at the 
CERN SPS, surprised many who expected the hot system to become more
dilute and more weakly-interacting at higher energies.

Collective flow was a major piece of the RHIC program, and its characterization
in terms of hydrodynamics was a crucial piece of evidence in the RHIC discovery
of the strongly-coupled quark gluon plasma (sQGP).  Crucial aspects of collective
flow, on the experimental and theoretical sides at RHIC, have been:
\begin{itemize}
\item Based on theoretical calculations, the average elliptic flow, scaled by the eccentricity, 
is thought to reach a limiting value in the limit where
the viscosity can be ignored.  RHIC data achieved this limit both integrally (integrated over \pT)
and for $v_2(\pT)$, which rises linearly until viscous corrections become large.
\item When studied as a function of particle type, it is found that heavier particles show
a smaller $v_2$ at the same \pT at low \pT, while this hierarchy flips above 1.5 GeV, where protons
typically have a 50\% higher value of $v_2$.  This behavior has been explained by the hadronization of
the system via constituent quarks which recombine into baryons with $v_2$(baryon)=$(3/2) \times v_2$(meson).
\item The deviations from ideal behavior can be systematically calculated by viscous corrections, and
all RHIC data point to a small but significant value of the ratio of shear viscosity to entropy density ($\eta/s$).
\item While difficult to calculate in the strongly-coupled limit, where the approximations required for kinetic theory break down, AdS/CFT-based calculations have shown that a wide range of strongly-coupled systems have a lower-bound on 
$\eta/s = 1/4\pi$.  The RHIC experimental data suggest values of 1-2 times this bound, although esimates are limited
by theoretical uncertainties related to the modeling of the initial state.
\item When studying smaller systems (particularly Cu+Cu), it was found that accounting for the event-by-event position of the nucleons in the nuclear wave functions showed scaling in the quantity $v_2/\epsilon$ between Cu+Cu and Au+Au, when
plotted as a function of the transverse density of charged particles at mid-rapidity, estimated by $dN_{ch}/d\eta/S$,
where $S$ is the overlap area of the two nuclei.
\end{itemize}

\subsection{Methods}
\label{subsect:pas:flow:methods}
Harmonic flow is a global modulation of essentially all of the particles in an event relative to an event plane 
appropriate to each harmonic.  However, there are additional sources of multiparticle correlations, some which 
lead to global correlations (momentum conservation) and others which only lead to correlations local in angular space
(e.g. resonance decays and jets).  Thus, care must be taken to minimize such ``non flow'' correlations.

One of the earliest methods for measuring flow was the ``event plane'' method, which calculates an event plane using
a forward detector and correlates all particles with this event plane, based on the so-called ``Q-vector'' for each order $n$:
\begin{equation}
\end{equation}
From this, the $n$-th order event plane $\Psi_n$ is simply determined as the angle of the Q-vector itself.  
From the definition of the Q-vector, the angle is $n$-fold ambiguous, but this has no effect on the extracted
$v_n$ coefficient, which is defined as 
\begin{equation}
v_n = \frac{v^{obs}_n}/\mathrm{Res}\{ n \Psi_n \} = \frac{ \langle \cos ( n [\phi - \Psi_n] ) \rangle }{ \langle \cos ( n [\Phi_n - \Psi_n] ) \rangle    }
\end{equation}
where $\Phi_n$ is the direction of the true event plane.  The latter quantity cannot be observed directly and so the 
resolution parameter must be derived from comparison of different detector regions.  This is typically done by comparing
symmetric pseudorapidity regions separated from the region being measured:
\begin{equation}
\mathrm{Res} \{ n \Psi_n^{\mathrm{P(N)}} \} =
\langle \cos n ( \Psi_n^{\mathrm{P(N)}} - \Psi_n ) \rangle = 
\sqrt{ \langle \cos n ( \Psi_n^{\mathrm{P}} - \Psi_n^{\mathrm{N}} ) \rangle }
\end{equation}
Non-flow contamination is typically most difficult to control when the subevents used to determine the event plane
and resolution are close in $\eta$ to each other or close to the measured particles.
%
Another method is to use multiparticle cumulants to explicitly remove lower order correlations.
These can be calculated either from a generating function formalism, or through moments of the Q-vector itself.
They are more or less sensitive to non-flow depending on the order of the cumulant.  For example, the two-particle
cumulant is quite senstitive to effects from resonance decay and jet fragmentation.  However, the four particle
cumulants are generally much less so, since it explicitly removes short-range two particle correlations.
%
A third method is to use the ``Lee Yang zeroes'' approach, which accounts for correlations of all lower orders using a
different generating function.  The method relates the zeros of a complex function to the magnitude of the relevant
flow harmonics.  While the method is thought to be robust against most sources of non-flow, it is particularly sensitive to
multiplicity fluctuations, and so is calculated using event samples with similar multiplicities within a given 
centrality interval.  Theses subsamples are then averged within the desired centrality interval to give the final result.

\subsection{Elliptic flow}
\begin{figure}[!tb]
\begin{center}
\includegraphics[height=0.49\textwidth]{flowcorrelations_figs/atlas_v2_fig_02.pdf}
\includegraphics[height=0.49\textwidth]{flowcorrelations_figs/fig2.pdf}
\caption[]{(left) ATLAS data showing the evolution of anisotropy relative to the reaction plane, as a function of centrality~\cite{ATLAS:2011ah} (right) First ALICE data on \vtwo in Pb+Pb collisions at the LHC~\cite{Aamodt:2010pa}.}
\label{fig:pas:fc:firstresults}
\end{center}
\end{figure}
The first LHC data on elliptic flow was released by the ALICE collaboration soon after the first collisions, 
and is shown in Figure~\ref{fig:pas:fc:firstresults}.
The elliptic flow was estimated using three methods: 2-particle cumulants ($v_2{2}$), 4-particle cumulants ($v_2{4}$)
and Lee-Yang Zeros (LYZ).  The first method is known to be sensitive to correlation from jets, which have 
a larger contribution at the LHC than at RHIC, and the latter method was only used for integral flow.  
The integral flow was found to be larger than that measured at RHIC, but only by about 15-20\%.
What was surprising was that, as a function of \pT (but only measured out to 4 GeV)
the magnitude of $v_2$ was found to be quantitatively very similar to that measured in
the STAR experiment at RHIC, using similar cumulant methods.
Although a similar scaling has been observed in the very low energy 
data on inclusive $v_2(\pT)$ from STAR taken during the recent RHIC energy scan,
there is no fundamental understanding yet of how this scaling arises.
It suggests that most of the variation in the integral elliptic flow results from the change in the spectral shape
of inclusive hadrons.
A hydrodynamic calculation by Luzum was able to reproduce this result soon after its release, 
based on a scaling of the initial energy density according to the measured charged-particle
multiplicity, and assuming no change in the medium transport properties~\cite{Luzum:2010ag}.
The first $v_2$ measurement at higher \pT was performed by ATLAS~\cite{ATLAS:2011ah}, 
showing the transition from the low \pT behavior,
understood by viscous hydrodynamics, to the higher \pT values presumably explained by the path-length dependence of
energy loss of jets.
By comparison with PHENIX data on $\pi^0$ particles, this shows that the scaling of $v_2$ extends to high \pT as well,
within the large statistical errors of the lower-energy measurement.

\begin{figure}[!tb]
\begin{center}
\includegraphics[width=0.55\textwidth]{flowcorrelations_figs/atlas_v2_fig_06.pdf}
\raisebox{.5cm}{\includegraphics[width=0.39\textwidth]{flowcorrelations_figs/v2eps_dNdetaoverS_PHOBOS.pdf}}
\caption[]{(left) ATLAS data showing the invariance of $\vtwo( \pT )$
  with beam energy~\cite{ATLAS:2011ah} (right) CMS compilation showing
  the observed scaling of $\vtwo /\epsilon$ vs. $(1/S)
  dN_{ch}/d\eta$~\cite{Chatrchyan:2012ta}.}
\label{fig:pas:fc:scaling}
\end{center}
\end{figure}
The dependence of the inclusive elliptic flow on the initial state geometry was studied carefully by CMS, who 
performed a careful extrapolation of the integral $v_2$ down to $\pT>0$ GeV, using simultaneous measurements of
$dN/d\pT$, to match with the lower energy PHOBOS data.
To factor out the initial shape and size of the overlap region, a Monte Carlo Glauber model was used to match
the centrality selections made with the CMS forward calorimeter.  
From these, the eccentricity and overlap area were calculated according to the definitions from 
subsection\ref{subsect:pas:flow:methods}.  The CMS data, shown superimposed on data from RHIC, is shown in 
Figure~\ref{fig:pas:fc:scaling}(right), and overlaps the PHOBOS data in the most peripheral collisions and
shows a continuous rise in the more central collisions, except perhaps in the most central interval.

\begin{figure}[!tb]
\begin{center}
\includegraphics[width=0.8\textwidth]{flowcorrelations_figs/v2_pt_12cen_4methods.pdf}
\caption[]{
CMS data showing $\vtwo(\pT)$ in centrality intervals, using four different methods of extracting
$\vtwo$: event plane (EP), 2-particle cumulants, 4-particle cumulants, and Lee-Yang Zeros~\cite{Chatrchyan:2012ta}.
}
\label{fig:pas:fc:methods}
\end{center}
\end{figure}

The CMS data shown in Figure~\ref{fig:pas:fc:methods} illustrates the $\pT$ and centrality dependence of
$v_2(\pT)$, comparing directly the different methods of flow reconstruction used for $v_2$.  
While it is clear that the 2-particle cumulant is the most contaminated by non-flow, particularly at high $\pT$,
one also observes some systematic differences between all three methods, particularly where the flow
is the strongest.

\begin{figure}[!tb]
\begin{center}
\includegraphics[width=0.35\textwidth]{flowcorrelations_figs/alice_fig5_vn_pid.pdf}
\includegraphics[width=0.63\textwidth]{flowcorrelations_figs/v2_pt_ep_atlas_alice_eta0-1_band_v5.pdf}
\caption[]{(left) ALICE data showing $\vtwo(\pT)$ for identified
  hadrons, for $|\eta|<0.8$~\cite{Abelev:2012di} (right) CMS data
  showing the $\vtwo$ for unidentified hadrons at very high \pT, out
  to 50 GeV~\cite{Chatrchyan:2012xq}}
\label{fig:pas:fc:highpt}
\end{center}
\end{figure}
The previous results have all been for unidentified hadrons.  This choice provides the largest phase
space coverage (particularly in $\pT$) and allows for comparisons between experiments with different particle
identification capabilities.  However, a study separating the different hadron species is crucial, 
given the previous measurements at RHIC showing the strong differences between them.
The ALICE data showing identified particles (separated using the $dE/dx$ measured in the ALICE TPC) for $\pT>3$ GeV
is shown in Figure~\ref{fig:pas:fc:highpt}(left).  The charged pion data on $v_2(\pT)$ is quantitatively similar to
the PHENIX $\pi^0$ data over the $\pT$ range where they overlap.  The proton-antiproton $v_2$ are found to be
subsantially larger than the pion values over the measured $\pT$ range, although it is clear that the peak-plateau
structure seen in the inclusive hadron data is not explained primarily by one particular hadron type.
Also, at the highest $\pT$ measured, the protons and pions are quite close, although the protons remain
systematically higher out to 14 GeV.
CMS extends the hadron $\pT$ range out to the full range provided by the LHC in 2011, using a high $\pT$
high level track trigger.  The data, shown in Figure~\ref{fig:pas:fc:highpt}(right), show that the 
plateau observed setting in above 6 GeV extends out to 50 GeV, with only mild decreases observed within
the stated uncertainties.

\begin{figure}[!tb]
\begin{center}
\includegraphics[width=0.98\textwidth]{flowcorrelations_figs/v2_etashifted_3cen_PHOBOS.pdf}
\caption[]{
CMS data showing $\vtwo$ for unidentified hadrons as a function of $\eta - y_{\mathrm{beam}}$, averaged over $0 <\pT < 3$ GeV (using an extrapolation procedure to cover $\pT <300$ Mev).
Results are compared to data from $\sqrt{s_{NN}}=200$ GeV
at large $\eta$ from the PHOBOS experiment at RHIC~\cite{Chatrchyan:2012ta}.
}
\label{fig:pas:fc:limfrag}
\end{center}
\end{figure}
Another intriguing way to compare with the lower energy data is suggested by studies performed by the PHOBOS
collaboration, which found that the integrated $v_2(\eta)$ is the same for different colliding beam energies,
when plotted in the rest frame of one of the projectiles.  This is the phenomenon of so-called
``limiting fragmentation'', where many quantities are found to depend only on their rapidity relative to either
beam or projectile.  
Figure~\ref{fig:pas:fc:limfrag} shows $v_2(\eta)$ as a function of $\eta - y_{\mathrm{beam}}$ (in the forward
hemisphere) and $\eta + y_{\mathrm{beam}}$ in the backward hemisphere.
While the CMS and PHOBOS data points do not overlap in any measured region, they appear to be continuous
in the forward LHC kinematics and the mid-rapidity RHIC kinematics.
However, it is clear that the behavior of the CMS data is much more suggestive of a boost-invariant central
plateau, while the PHOBOS data did not show similar behavior.

\subsection{Higher order harmonics}

\begin{figure}[!tb]
\begin{center}
\includegraphics[height=0.6\textwidth]{flowcorrelations_figs/atlas_vn_fig_04.pdf}
\includegraphics[height=0.6\textwidth]{flowcorrelations_figs/atlas_vn_fig_05.pdf}
\caption[]{(left) ATLAS data showing $v_n(\pT)$ for different centrality intervals and $|\eta|<2.5$, for $n=2-6$.  Very little dependence on $\eta$ is observed~\cite{ATLAS:2012at}.  
(right) The same ATLAS data, in \pT intervals, showing the centrality dependence of $v_n$~\cite{ATLAS:2012at}.}
\label{fig:pas:fc:vn}
\end{center}
\end{figure}
Although the realization that $v_2$ was sensitive to fluctuations in the nuclear overlap, particularly from
the eventwise random positions of nucleons in each nuclei, several more years elapsed before it was suggested
to look for higher-order harmonic flow, particuarly odd-orders~\cite{Alver:2010gr}.  
Many authors argued that symmetric systems
would have a zero $v_1$, $v_3$, $v_5$, etc, but had not yet considered the effect of fluctuations.
By mid-2011, all of the LHC and RHIC experiments had significant measureents of many of the higher-order
contributions, up to $v_6$ which was the limit of the statistics in the 2010 run.

CMS released the first evidence for the presence of higher-order harmonics in the two-particle
correlation function.  ALICE released the first measurements of of $v_3-v_5$ up to $\pT=4$ GeV,
using two and four particle cumulants~\cite{ALICE:2011ab}, and followed up with an extensive study of two-particle
correlations in Ref.~\cite{Aamodt:2011by}.
ATLAS released the first large scale study of higher harmonics
for unidentified hadrons, with the first complete experimental measurements of

The ATLAS data shown in Figure~\ref{fig:pas:fc:vn}(left) and (right) show the harmonics $v_2 - v_6$ as
a function of \pT in centrality intervals from the most central 0-5\% to the most peripheral 70-80\%.
The the figure on the left shows that the pattern is consistently the same for all harmonics: 
a rapid rise starting at low \pT, a peak around 3 GeV, and a
rapid decrease out to higher $\pT$.  While all of the experiments have demonstrated that $v_2$ does not 
necessarily go to zero at high $\pT$, it remains an open question about the higher harmonics, which can
only be resolved with higher statistics.
The right figure shows the centrality dependence in small \pT intervals, demonstrating that $v_2$ is fundamentally
different than the higher harmonics, having a much milder centrality dependence.  While $v_2$ mainly reflects
the overall geometric shape of the system, the higher harmonics seem to mainly reflect fluctuations.
However, the decrease in magnitude with increasing $n$ is generally thought to reflect the presence of viscous effects.

\begin{figure}[!tb]
\begin{center}
\includegraphics[width=0.98\textwidth]{flowcorrelations_figs/atlas_vn_fig_13.pdf}
\caption[]{
ATLAS data showing the $n$ dependence of $v_n$ in four \pT intervals, which are effectively
angular power spectra at different resolution scales.
}
\label{fig:pas:fc:powerspec}
\end{center}
\end{figure}
Another representation of this data is given in Figure~\ref{fig:pas:fc:powerspec}, which shows the ``angular power spectrum'',
the $n$ dependence of $v_2(\pT)$ for particular intervals in \pT and centrality.  The fall-off with increasing $n$ is a general
feature, of the data which is expected to be driven by viscous corrections that increase with $n$.

\begin{figure}[!tb]
\begin{center}
\includegraphics[width=0.98\textwidth]{flowcorrelations_figs/atlas_vn_fig_20a.pdf}
\includegraphics[width=0.93\textwidth]{flowcorrelations_figs/atlas_vn_fig_21.pdf}
\caption[]{ (top) ATLAS data showing the amplitude of $v_{1,1}$ the
  dipole modulation in the 2-particle correlation function, as a
  function of $p^b_T$ for ranges in $p^a_T$.  The fit used to extract
  the functional form of $v_1(\pT)$ is shown~\cite{ATLAS:2012at}.
  (bottom) The extracted functional form of $v_1(\pT)$, from the fits
  shown above, as a function of centrality~\cite{ATLAS:2012at}.  }
\label{fig:pas:fc:v1}
\end{center}
\end{figure}
While the two-particle analyses from ALICE, ATLAS and CMS all utilized a $v_{1,1}$ term in their fits, the first measurements
did not extract a single-particle $v_1$ contribution, since momentum conservation is expected to be a substantial effect.
The first published extraction of $v_1$ from experimental data was performed by a theoretical team using published ALICE data,
doing a fit of the form $V_{1\Delta}(p^a_T,p^b_T) = v_1(p^a_T)v_1(p^b_T) - k p^a_T p^b_T$.
ATLAS was the first experimental group to measure directed flow from its own data, using the same approach 
as Ref.~\cite{Retinskaya:2012ky}(but performed in isolation from it), fitting $v_1(\pT)$
The value of $v_1$ was also found to change very little with centrality, suggesting it too arises from fluctuations.

\subsection{Flow fluctuations}
\begin{figure}[!tb]
\begin{center}
\includegraphics[width=0.98\textwidth]{flowcorrelations_figs/atlas_v2fluc_fig_10.pdf}
\caption[]{
ATLAS data on the event-wise distributions of the harmonic coefficients $v_2 - v_4$ presented in a selection of
centrality intervals, from Ref.~\cite{Aad:2013xma}.
}
\label{fig:pas:fc:flowfluc1}
\end{center}
\end{figure}
\begin{figure}[!tb]
\begin{center}
\includegraphics[width=0.40\textwidth]{flowcorrelations_figs/vnSigmaPrelim.pdf}
\raisebox{0.5cm}{\includegraphics[width=0.55\textwidth]{flowcorrelations_figs/fig3_v2_sigma.pdf}}
\caption[]{
(left) CMS data on the centrality dependence of the standard deviation of $v_2$ 
divided by the mean, for $0.3 < \pT < 3$ GeV,
extracted via the differences between the EP and cumulant results~\cite{Chatrchyan:2013kba}.
(right) ALICE data on the $\pT$ dependence of the same quantity, out to $\pT=8$ GeV, 
for a selected set of centrality
intervals~\cite{Abelev:2012di}.
}
\label{fig:pas:fc:flowfluc2}
\end{center}
\end{figure}

The same geometric fluctuations that lead to the presence of the higher harmonics are also expected to lead to 
event-by-event fluctuations in the individual coefficents.  This has been studied using two methods.
ATLAS developed a data-driven method to unfold the measured $v_n$ distributions ($P(v_n)$) with a Bayesian 
technique.  The distributions $P(v_n)$ are shown in Figure~\ref{fig:pas:fc:flowfluc1} for $v_2$, $v_3$ and $v_4$
for a selected set of centrality intervals.  It is clear that the fluctuations are large in all selected samples.
Although the distributions are not Gaussian, but are rather projections of a 2D Gaussian (also known as a
Bessel-Gaussian distribution),
the distributions are typically quantified using the first two moments, the mean and standard deviation.
These can also be estimated by combining different estimates of $v_n$, in particular using the event-plane 
and 4-particle cumulant methods.  Based on the analysis of Ref.{}, 
the difference between these quantities is twice the variance, while their sum is twice the squared mean.
Figure~\ref{fig:pas:fc:flowfluc2}(left) shows CMS data on $\sigma/\langle v_2 \rangle$ derived 
from cumulants, compared with the ATLAS data derived from the fully unfolded distributions.
While the $v_2$ fluctuations compare well between the two methods, the ATLAS and CMS results on $v_3$
are quite different, possibly from the inapplicability of the cumulant approach for this quantity.
The ATLAS results are essentially constant for all centrality intervals at around 0.5, which is the 
value one gets ($\sqrt{4/\pi-1}$) if the fluctuations are described by a 2D Gaussian, 
projected along the radial direction.
Figure~\ref{fig:pas:fc:flowfluc1}(left) shows ALICE data on $\sigma/\langle v_2 \rangle$ derived 
using a similar approach as CMS.  For centrality intervals from 5-60\%, the overall magnitude
is similar to that seen by the other experiments.  However, this figure points out that the fluctuations
have essentially no centrality dependence out to moderately large \pT, about 8 GeV, which reaches into 
the plateau region typically associated with differential energy loss.




