\section{Flow correlations}
\label{sec:ps:flow}
%5. Flow Correlations & Flow & PID Flow (PS)
%- Geometric fluctuations and origin of v_n vs ecc_n (Glauber, v3)
%- non-PID flow (ALICE first results)
%- PID flow (ALICE v2 scaling)
%- 2PC (CMS v_n?)
%- Event plane (ATLAS EP)
%- Explaining the ridge (2PC from 3 experiments)
%- Flow fluctuations
%- Event plane correlations
%- Success of viscous hydro
%- Comparison with RHIC
%- Flow in p+Pb (ridge, double ridge, PID)
%- Open questions

While the previous section covered physical mechanisms which induce
correlations between multiple hadrons, this section covers the
phenomenon of ``collective flow'', which leads to the correlations of
essentially all the particles in every event.
This results from the translation of anisotropies in the initial shape of the
colliding nuclei into anisotropies in momentum space, something that
would not occur if individual nucleon-nucleon collisions emitted independently
of each other.
The characterization of a ``shape'' in a final state particle distribution
is typically performed using a Fourier decomposition of the azimuthual
angle distrbution of final state particles.
Of course, averaging over an ensemble of independent events would lead to
the observation of no net anisotropy.  Thus, the presence of harmonic
oscillations in the final state requires the estimation of an ``event plane''
from the particle themselves, with an axis that points in the direction of
the largest momentum flow.

While this phenomenon was observed decades ago in the collisions of large
nuclei at low energies, this was straightforward to understand as the
reinteraction of the initial baryons and the produced hadrons, which
would thermalize and evolve as a ``hadron gas''.  
However, its persistence at higher energies, particularly 
\begin{figure}[!htb]
\begin{center}
\includegraphics[width=0.49\textwidth]{flowcorrelations_figs/atlas_v2_fig_02.pdf}
\includegraphics[width=0.49\textwidth]{flowcorrelations_figs/fig2.pdf}
\caption[]{(left) ATLAS data showing the evolution of anisotropy relative to the reaction plane, as a function of centrality (right) First ALICE data on \vtwo in Pb+Pb collisions at the LHC.}
\label{fig:pas:fc:firstreusults}
\end{center}
\end{figure}

\begin{figure}[!htb]
\begin{center}
\includegraphics[width=0.49\textwidth]{flowcorrelations_figs/atlas_v2_fig_06.pdf}
\includegraphics[width=0.49\textwidth]{flowcorrelations_figs/v2eps_dNdetaoverS_PHOBOS.pdf}
\caption[]{(left) ATLAS data showing the invariance of $\vtwo( \pT )$ with beam energy (right) CMS compilation showing the observed scaling of $\vtwo /\epsilon$ vs. $(1/S) dN_{ch}/d\eta$.}
\label{fig:pas:scaling}
\end{center}
\end{figure}

\begin{figure}[!htb]
\begin{center}
\includegraphics[width=0.49\textwidth]{flowcorrelations_figs/fig5_vn_pid.pdf}
\includegraphics[width=0.49\textwidth]{flowcorrelations_figs/v2_pt_ep_atlas_alice_eta0-1_band_v5.pdf}
\caption[]{(left) ALICE data showing $\vtwo(\pT)$ for identified hadrons, for $|\eta|<0.8$ (right) CMS data showing the $\vtwo$ for unidentified hadrons at very high \pT, out to 50 GeV}
\label{fig:pas:fc:highpt}
\end{center}
\end{figure}

\begin{figure}[!htb]
\begin{center}
\includegraphics[width=0.49\textwidth]{flowcorrelations_figs/atlas_vn_fig_04.pdf}
\includegraphics[width=0.49\textwidth]{flowcorrelations_figs/atlas_vn_fig_05.pdf}
\caption[]{(left) ATLAS data showing $v_n(\pT)$ for different centrality intervals and $|\eta|<2.5$, for $n=2-6$.  Very little dependence on $\eta$ is observed.  
(right) The same ATLAS data, in \pT intervals, showing the centrality dependence of $v_n$.}
\label{fig:pas:fc:vn}
\end{center}
\end{figure}

\begin{figure}[!htb]
\begin{center}
\includegraphics[width=0.98\textwidth]{flowcorrelations_figs/atlas_vn_fig_13.pdf}
\caption[]{
ATLAS data showing the $n$ dependence of $v_n$ in four \pT intervals, which are effectively
angular power spectra at different resolution scales.
}
\label{fig:pas:fc:powerspec}
\end{center}
\end{figure}

\begin{figure}[!htb]
\begin{center}
\includegraphics[width=0.98\textwidth]{flowcorrelations_figs/atlas_vn_fig_20a.pdf}
\includegraphics[width=0.98\textwidth]{flowcorrelations_figs/atlas_vn_fig_21.pdf}
\caption[]{
(top) ATLAS data showing the amplitude of $v_{1,1}$ the dipole modulation in the 2-particle correlation function, as a function of $p^b_T$ for ranges in $p^a_T$.  The fit used to extract the functional form of $v_1(\pT)$ is shown.
(bottom) The extracted functional form of $v_1(\pT)$, from the fits shown above, as a function of centrality.
}
\label{fig:pas:fc:v1}
\end{center}
\end{figure}

\begin{figure}[!htb]
\begin{center}
\includegraphics[width=0.98\textwidth]{flowcorrelations_figs/v2_etashifted_3cen_PHOBOS.pdf}
\caption[]{
CMS data showing $\vtwo$ for unidentified hadrons as a function of $\eta - y_{\mathrm{beam}}$, averaged over $0 <\pT < 3$ GeV (using an extrapolation procedure to cover $\pT <300$ Mev).
Results are compared to data from $\sqrt{s_{NN}}=200$ GeV
at large $\eta$ from the PHOBOS experiment at RHIC.
}
\label{fig:pas:fc:limfrag}
\end{center}
\end{figure}

\begin{figure}[!htb]
\begin{center}
\includegraphics[width=0.98\textwidth]{flowcorrelations_figs/v2_pt_12cen_4methods.pdf}
\caption[]{
CMS data showing $\vtwo(\pT)$ in centrality intervals, using four different methods of extracting
$\vtwo$: event plane (EP), 2-particle cumulants, 4-particle cumulants, and Lee-Yang Zeros.
}
\label{fig:pas:fc:methods}
\end{center}
\end{figure}



