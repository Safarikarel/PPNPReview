\section{Summary and outlook}
\label{secall:summary}
This review has attempted to summarize the broad range of new data from the first two
runs with lead ions in collision at the CERN Large Hadron Collider.
To date, about 150~$\mu$b$-1$ of data have been collected by each of
three major LHC experiments (ALICE, ATLAS and CMS), corresponding to about three billion
minimum-bias events sampled between them.
The three detectors are all large, multi-purpose spectrometers, with ALICE being optimized
for particle identification at low $p_{\rm T}$, photons, and muons at forward rapidity;
ATLAS and CMS optimized for precision measurements
of high \pT\ objects (tracks, jets, muons, electrons and photons).
The centrality of the collision is estimated in all of the experiments by partitioning
the event distribution of the forward transverse energy or multiplicity into
intervals corresponding to percentiles of the collected events, and related to
the nuclear overlap geometry using Glauber models.
Good agreement in the centrality estimation is found between the experiments, as illustrated
by the charged particle multiplicity per participant pair, as a function of centrality.
The centrality is also crucial for measuring the suppression of high \pT hadrons, relative
to what is measured in proton-proton collisions, via the ratio $R_{\mathrm{AA}}( \pT )$.
The value of $R_{\mathrm{AA}}$ is found to be strongly suppressed at low \pT, but then to rise
and plateau again at a level of about 0.5, similar to what is found for jet suppression.
The detailed study of identified particle yields shows good consistency with thermal models,
except for an anomalous suppression of protons.
Just as it was at the CERN SPS and RHIC, the yield of strange hadrons, and especially multi-strange, is substantially
enhanced in heavy-ion collisions.
Two particle correlations provide a rich handle on soft physics in heavy ions, giving
insight into the space--time structure (via HBT measurements), jets, and the so-called
``ridge''.  Despite not being a full jet measurement, correlations provide similar insight
into the suppression of the away-side region and an enhancement on the near-side
suggestive of modified jet fragmentation in medium.
The geometric dependence of charge fluctuations and the charged balance function provide
information on the fate of conserved quantities in the medium evolution.
Finally, more sophisticated correlations meant to look for the Chiral Magnetic Effect (CME)
find consistency with STAR data from RHIC, but no clear indication of the CME.

Collective flow has been extensively measured at the LHC by all the experiments.
Both two-particle and multi-particle methods have been used, which have varying sensitivity to
the presence of non-flow effects, such as correlations from jet production and fragmentation.
Various scaling behaviors have been observed, both at the absolute value of $v_2$ even at
large \pT, but also the $v_2$ scaled by the overlap shape.
The LHC has also seen an explosion in the measurement of higher-order harmonics up to $v_6$,
and the rapidity-even and -odd contributions to $v_1$, all of which should
allow more stringent tests of hydrodynamic models.
Event-by-event flow fluctuations have also been measured both by unfolding reconstructed
distributions as well as by using combinations of cumulants.  Good consistency has been found
for $v_2$ fluctuations over a wide range of centrality, but with some differences in $v_3$
fluctuations.

Hard processes have been well calibrated using measurements of electro-weak bosons: both
$W$ and $Z$ as well as photons.  All of these particles are found to have cross sections
(both total and differential)
consistent with perturbative QCD calculations scaled by the number of binary collisions.
No substantial modifications of the nuclear parton distributions functions appears to be needed, due to the good
agreement of rapidity-dependent quantities.
In this context, the modifications of jets in more central heavy ion collisions can more
clearly be attributed to the energy loss of partons traversing the hot, dense medium.
Energy loss has been addressed using several techniques, from dijet imbalance (despite measurements
showing full energy containment) to
single jet suppression, both inclusive and differentially in $\varphi$.  Jet fragmentation
in heavy-ion collisions has also been measured and found to be substantially modified, especially at
large angles with respect to the jet axis.
Correlations of jets with photons shows a similar energy loss effect, using a process that allows
tagging of the initial hardness scale.

Heavy flavor is produced copiously at the LHC and is expected to show different energy loss than
light quarks.  Early results show strong suppression of D mesons as well as J$/\Psi$ from
B-meson decays, as well as a significant signal that D mesons participate in the collective flow.
Quarkonia (both J$/\psi$ and $\Upsilon$ states) have also been studied over a wide range of \pT\
and rapidity.  The J$/\psi$ yield is found to be suppressed at a similar level over a wide
rapidity range, although the suppression at low $\pT$ is not as strong as it was at
RHIC, suggesting the possibility of regeneration processes due to the large charm quark production
rates.  Similar to what was found for open charm, the forward J$/\psi$ have a
$2\sigma$ significant $v_2$ signal, potentially larger than was found at RHIC.
The $\Upsilon$ family show interesting behavior in heavy ion collisions, with all states being
suppressed relative to proton-proton collisions, and the more weakly bound higher states showing
stronger suppression than the more tightly bound 1S state.

The results shown here are mainly the ones submitted to journals for publication, and many more
results are in preparation.  Thus, this review should be seen mainly as a snapshot of
the state of the field.  The wealth of upcoming results from the first two lead--lead runs,
and the upcoming runs with higher energy ($\energy = 5.5$ TeV) and luminosity
exceeding the design instantaneous luminosity of $10^{27}$~cm$^{-2}$s$^{-1}$ should
provide even further insight into
the nature and properties of the hot, dense matter formed in nuclear collisions at the LHC.

The progress towards the detail characterization of this strongly interacting state of matter will now focus on rare probes, and the study of their coupling with the medium and hadronization modes. These will include heavy-flavour particles, quarkonia states, real and virtual photons, jets and their correlations with other probes. The cross sections of all these processes are significantly larger at LHC than at previous accelerators. In addition, the interaction with the medium of such hard probes is better controlled theoretically than the propagation of light partons.

To achieve these goals high statistics and high precision measurements are required, which will give access to the rare physics channels needed to understand the dynamics of this condensed phase of QCD.  Therefore, the LHC collaborations are planning to upgrade the current detectors by enhancing their vertexing capability, and allowing data taking at substantially higher rates.
The upgrade strategy for heavy-ion running is formulated under the assumption that, after the second long shutdown in 2018, the LHC will progressively increase its luminosity with Pb beams eventually reaching an interaction rate of about 50~kHz, i.e. instantaneous luminosities $6 \times 10^{27}$~cm$^{-2}$s$^{-1}$. The proposed plan~\cite{ALICEUpgradeLoI} envisage  to accumulate 10~nb$^{-1}$ of Pb--Pb collisions inspecting ${\cal O}(10^{10})$ interactions, which is needed to address the proposed physics programme, with focus on rare probes both at low- and high-transverse momenta as well as on the multi-dimensional analysis of such probes with respect to centrality, event plane, and multi-particle correlations.

